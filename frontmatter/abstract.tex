%!TEX root = ../thesis.tex
\chapter*{概要}
\thispagestyle{empty}
%
\begin{center}
  \scalebox{1.5}{自律移動ロボット用の雨天シミュレータの開発}\\
  \scalebox{1.5}{(検出された雨粒の時間間隔のモデル化)}
  \vspace{1.0zh}
\end{center}
\vspace{1.0zh}
%

%
近年,自律移動ロボットは幅広い産業分野で需要が高まっており屋外で雨天においても自律移動できることが望まれている.
本研究室では,2D LiDARを搭載した屋外自律移動ロボットを研究・開発している.
しかし,雨天環境での自律移動はLiDARが雨粒を検出し停止,回避動作を行うことにより困難である.
先行研究\cite{mura}では雨天のLiDARデータを取得し,取得したデータを解析,モデル化した.
しかし,取得したデータが少ないことや雨粒の時間間隔のモデル化ができていない.
本論文では,先行研究で不足する雨天時のLiDARデータの追加取得,雨粒の時間間隔のモデル化を行い,自律移動ロボット用の雨天シミュレータの開発を行った.
シミュレータでは雨粒の距離のモデル化に基づいた雨を再現することができた.
%
%
%
%
%
%
%
%
%
%
%
%
%

\vspace{2.0zh}
キーワード: 2DLiDAR,雨天,シミュレータ
%

\newpage
%%
\chapter*{abstract}
\thispagestyle{empty}
%
\begin{center}
  \scalebox{1.3}{Development of a Rainy Weather Simulator for Autonomous Mobile Robots}
  \scalebox{1.3}{(Modeling the Temporal Intervals of Detected Raindrops)}
\end{center}
\vspace{1.0zh}
%
In recent years, the demand for autonomous mobile robots has been increasing across various industrial sectors, with the capability to operate autonomously outdoors even in rainy weather being highly desirable.  
In our laboratory, we are engaged in the research and development of outdoor autonomous mobile robots equipped with 2D LiDAR.
However, achieving autonomous navigation in rainy conditions is challenging due to LiDAR detecting raindrops, leading to unintended stops or avoidance maneuvers.  
In prior research by Murabayashi [1], LiDAR data in rainy conditions was collected, analyzed, and modeled. 
However, the amount of collected data was insufficient, and the temporal modeling of raindrops was not addressed.  
This paper focuses on supplementing the insufficient rainy weather LiDAR data from previous studies and modeling the temporal intervals of raindrops.
Based on these efforts, a rainy weather simulator for autonomous mobile robots was developed.
The simulator successfully reproduces rainfall based on the modeled distance of raindrops.
%
\vspace{2.0zh}

keywords: 2D LiDAR, rainyweather, simulator