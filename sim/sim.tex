先行研究\cite{mura}及び前章で得られた解析結果とモデル化を活用することで,雨天環境をシミュレータ上で再現可能であると考えられる.
本論文では,「検出された雨粒の距離の分布」に基づいて雨を再現した.
しかし,「検出された雨粒の時間間隔の分布」や「隣接するレーザーで検出された雨粒の距離の分布」については未だ組み込まれていない.
これらを追加することで,実環境により近い,様々な雨量の雨粒を再現できると期待される.

\section{シミュレータ環境}
シミュレータ環境はGazebo\cite{gaze}である.
本研究室で開発されたornebox simulator\cite{simu}をベースに開発を行った.
シミュレータ内のLiDARが検出する距離に雨を模したノイズを追加することで,雨天環境を再現した.

\section{シミュレーション結果}
「検出された雨粒の距離の分布」で得た平均値と標準偏差を基に雨を模したノイズを追加し雨天環境の再現を行った.
検出された雨粒の距離の分布を示す.